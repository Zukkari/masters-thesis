\TODO{
    Describe the tool here.
}

\TODO{
    What did we build?

    Plugin for SonarQube that can detect 29 code smells.
    We need a tool that can scan a large number of applications.
    This is best achieved when the tool can be run automatically for a given input project
    and since the corpus is large, the analysis should be performed on the server side by the program
    and not by the human who would perform a manual check.
}
\subsection{SonarQube plugin}\label{subsec:sonarqube-plugin}

In order to fulfil our task of analyzing a large corpus of application to detect the code smells,
we built a tool that is stable, scalable and allows us to aggregate the results of the
analysis in an organized manner.
Moreover, we needed a framework that would allow us to analyze a large corpus of applications programmatically since starting
the analysis manually for every project under observation would be inefficient and unproductive.

For this task, we decided to use the SonarQube platform because it is a de facto tool in the industry to use
for static analysis of the applications.
Not only that, but SonarQube provides possibilities to write custom rules by writing custom plugins.
Since we needed to implement code smells that are not yet defined by the SonarQube, we decided to extend
the tool by writing a plugin that can detect the code smells that are described in subsection~\ref{subsec:code-smells}.

\TODO{
    How?

    Followed tutorial that is available on SonarQube documentation page (https://docs.sonarqube.org/display/PLUG/Writing+Custom+Java+Rules+101).
    But since there were some issues (describe issues with classpath, describe how the analysis works), we had to
    reuse some of the internals of the SonaQube Java module.

    Here we also say that there are multiple contexts where the plugin runs.
    One of the contexts is to run the plugin on the server side, which is supposed
    to be run during CI/CD pipeline, and another context is to run inside developers IDE
    to provide instant feedback without the need to compile the code.
}

To create the plugin, we followed the tutorial provided in the SonarQube documentation~\cite{sonar_plugin_tutorial}.
The documentation provides guidelines on how to create a plugin with custom Java rules, how to test the plugin and
how to register rules with the SonarQube so that it would find them during runtime of the application.
The documentation relies on extension of Sonar Java plugin~\cite{sonar_java_plugin}, which provides an API
for the Java languages abstract syntax tree (AST) and basic interface to create rules, which would be used
during the analysis.

However, this tutorial only focuses on running on an instance of SonarQube and not SonarLint, which is an
extension to run the plugins inside the integrated development environment (IDE).
This is relevant because both SonarQube and SonarLint rely on Sonar compute engine, which means that you can write
a plugin for either of those tools and it would be usable in both of them.

During the runtime of SonarQube, plugins can be installed dynamically, either from the marketplace or they can
also loaded from the \url{/opt/sonarqube/extensions/plugins/} directory as stated in the documentation.
This means that plugins will be loaded dynamically during the execution of the analysis and since the documentation
states the the Sonar Java plugin must be included with \verb|provided| scope during compilation (which means that it
will not be included in the final compiled artifact of the plugin), it means that not all of the classes might be
available at the runtime that were available during compilation.

\TODO{
    What is the goal?

    The goal is to build a tool, that can help developers in static analysis of the code.
    The tool would detect the list of code smells found in the appendix.
    Previously we mentioned that there are 2 contexts in which the plugin can be run, and in
    this thesis we will focus only on the server side of the tool.
    This is because we are interested in analyzing a large corpus of the applications
    and this best done on the server side, because manually skimming through all of
    the detected code smells in the IDE is not an option when you have a corpus of 1000 applications.
}

Previously we mentioned that there are multiple contexts in which the analysis can be run (SonarQube and SonarLint),
however, in this thesis, we will only focus on SonarQube because we are interested in analysis of the large corpus
of applications and it makes no sense to do this manually through the IDE\@.

Thus, we implemented the plugin that detects the code smells described in subsection~\ref{subsec:code-smells} and
can be executed inside a SonarQube instance.

\TODO{
    What tools did we use?

    SonarQube as a platform to run the analysis - provides nice UI for the end users and also
    very mature tool in the industry.
    But most importantly, allows us to analyze the applications and controls the whole flow of the analysis
    (starting the analysis, running the analysis, creating results, uploading them and then displaying them to the user).
    So we only need to provide our custom rules and a way to load them.

    Scala - language that we used to write the rules in.
    Scala is a programming language that combines both functional and object oriented approaches.
    So this was a good choice for me, because I wanted to write the code in a functional way but
    SonarQube is written in Java and API is designed in an object oriented matter.
    So Scala provided nice interop with Java API because it supports both functional and
    imperative approaches like described previously.

    Sonar Java plugin - used this as a base, because it provides all necessary tools that are
    required to parse Java sources into the AST and also provides needed utilities during the analysis
    (detection of cognitive complexity, number of lines of code etc).
}

\subsection{Tools used}\label{subsec:tools-used}

In order to implement the plugin, we used SonarQube as the implementation platform.
We decided to go with SonarQube platform for two primary reasons.
Firstly, SonarQube is a very mature tool in the industry.
Secondly, it controls the flow of the analysis (project configuration, starting
the analysis, creating the results, serializing and parsing the results, uploading them
to the server and displays them to the user), so we can focus on writing the custom code smell
detection rules.
Finally, it provides other features such as a web API, which can be used for exporting the analysis results.

Scala was chosen as the language for the plugin implementation.
We chose Scala, because it combines features of both functional and object oriented languages and since SonarQube
API is written in Java, it allowed for nice interoperability between Java and Scala.
Moreover, Scala provides some features that are not available in Java natively, such as implicit classes and
method parameters, and pattern matching.

As a plugin implementation base, we used Sonar Java plugin since it is recommended
base when writing plugins for the Java language and as mentioned previously, it provides
an API to use Java AST\@.

\subsection{Encountered problems}\label{subsec:encountered-problems}

Previously we also mentioned that plugins are loaded into the SonarQube dynamically and that
this plugin must be used with \verb|provided| scope when compiling the plugin.
This means that only some of the packages are provided during the runtime and not all of the classes
provided by the Sonar Java plugin can be used during the runtime.
This is an issue, because the plugin itself provides a lot of utilities that could be reused inside
the custom rules (for example cognitive complexity counter, lines of code counter).
To overcome this limitation, we extracted those utilities into a separate package and included it in
the compiled artifact of our plugin, so that we could use those tools during the analysis execution.

\subsection{Testing}\label{subsec:testing}

\TODO{
    Describe how did we test it?

    Mostly unit tests, to test the internal architecture of the plugin.
    In order to check the rules validity, SonarQube provides a framework to
    analyze the files the same way that the scanner on the server side in production would do.

    So, the rules were validated by writing a code snippet that would be represents a given
    code smell that we want to detect in a given rule.
    Then, you can describe which line of the code is invalid (e.g. you want to detect classes that contain)
    a code smell pattern.
    Then, the code snippet would contain comment in pattern (provide pattern here and also an example).
}

Once the plugin was implemented, we need to also test it to test the operation of the plugin.
Testing of the plugin can be split into two phases: testing of plugin internal architecture, such as
rule loading, registration, rule metadata loading and testing of rule definitions and verifying that
the rule detects code smells if provided with a valid code snippet.

In order to satisfy the first phase of the testing, we used Scalatest~\cite{scalatest} to write basic tests and to verify
that plugins internal components operate correctly.
We chose Scalatest, because it is a standard unit testing framework for Scala projects and it allows developers
to write unit tests in form of specification.

For the second phase, SonarQube provides a testing framework that allows to test the rules in a similar fashion that
they would have been used in production.
In this framework, testing is performed as follows: firstly, you need to specify a rule which you would like to test by
creating an instance of the rule and passing it to the framework, and secondly, you need to provide a code snippet that
contains a code smell that you would like to verify with the rule.

We can see an example of such test file in figure~\ref{data_class_example}.
This code snippet contains code that a person writing test would consider a code smell for a particular rule
and as seen on line 3, we can specify which line the plugin should report by placing a comment on that line in form of line comment:
\verb|// Noncompliant {{Optional message}}|.
Then the framework will verify if the plugin reports an issue in the provided code snippet and whether the plugin reported
the same line as marked in the source file and if the message was the same as provided in the source file.
This approach allows us to verify the code smells without the need to run the analysis separately on the server and
detect most of the issues with the definitions during testing phase.

\begin{figure} [htb]
    \begin{lstlisting}
package com.example.test;

public class DataClass { // Noncompliant {{Refactor this class so it includes more than just data}}
        public String field1;
        public int field2;
    }
    \end{lstlisting}
    \caption{Example of a data class code smell test file.}
    \label{data_class_example}
\end{figure}
