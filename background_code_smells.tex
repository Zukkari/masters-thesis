
In the era of agile software development, the quality of the code produced by developers plays a very important role.
It is crucial because developers have to be able to change the code quickly due to evolving business requirements.
However, due to time constraints and constant changes in the codebase, the code may become less maintainable.
Often developers make changes that allow them to implement the current requirements quickly, but if the requirements
change in the future, they would have to refactor large portions of code that they developed earlier.
Applying solutions that may cause future code refactoring is called technical debt~\cite{technical-debt}.

Often technical debt comes in pieces that can be universally unified into recognizable patterns.
Such technical debt patterns that can be applied to most software applications are called code smells.
According to~\cite{code-smell-definition}, code smells are symptoms of poor design and implementation choices, which
lead to increased fault-proneness and, eventually, decrease software maintainability.
Decreased maintainability leads to costs when implementing new features or making changes to existing ones.

Since code smells come in patterns that can be recognized by humans, we could also teach machines to recognize those
patterns in the code that humans write.
The software that can perform static code analysis to recognize code smells and problematic patterns in the code
is called a static analyzer.
A static analyzer performs an analysis of the code without actually executing it~\cite{static-analyzers-paper}.

\citeauthor{refactoring-fowler} describes various code smells in his book~\cite{refactoring-fowler}.
The book describes 22 code smells and discusses how the code smells can be fixed by applying different
patterns.
Let's look into some examples of the code smells defined by~\citeauthor{refactoring-fowler}.

\begin{flushleft}
    \textbf{Data class} –- a class which only contains data and no business logic that interacts
    with the data.
    The class usually contains only public fields or private fields that only have getters and setters.
\end{flushleft}

\begin{flushleft}
    \textbf{Cyclic dependencies} -- classes that depend on each other.
    This is an issue because this makes dependency injection impossible, due to the fact that dependency graph forms
    a cycle, and it is not possible to determine which class should be instantiated first.
    Dependencies should be representable as an acyclic graph as defined by~\citeauthor{refactoring-fowler}.
\end{flushleft}

\begin{flushleft}
    \textbf{Feature envy} -- code smell where a class is interested in the data of the other classes
    more than its own data.
    Usually occurs when a class gets the data externally and tries to compute some value.
    Such approach goes against OOP principles, which require that a class should use its internal state to compute the values.
\end{flushleft}

In this paper, we implemented the code smells defined by~\citeauthor{refactoring-fowler}.
The full list of implemented code smells can be found in the appendix~\ref{subsec:implemented-code-smells}.
