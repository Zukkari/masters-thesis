SonarQube~\cite{sonar} is a platform that automates the process of code reviewing in order to detect code smells,
bugs and vulnerabilities in the source code.

SonarQube is often used as a stage in the CI/CD pipeline which sets a \textit{quality gate} for the developers.
Quality gate consists of \textit{rules} which are defined by the \textit{profile}.
A rule is a definition which describes a certain pattern in the source code (such as code smell, bug or a vulnerability) which
must be avoided as it might cause maintainability issues or introduce new bugs and vulnerabilities.
The set of rules that is used for a project can be configured individually for each project.

There are multiple reasons why we chose SonarQube as an implementation platform.
Firstly, SonarQube is one of the largest ecosystems that perform static code analysis for a wide variety of languages.
The framework comes with numerous predefined code smells that are very common in the industry.
SonarQube is open-source, has been in development since 2006 and has over 29,000 commits in its Github repository\cite{sonar-repo}.
Therefore, we can say that this framework is pretty mature.
Furthermore, SonarQube is also one of the most popular tools in industry to perform static code analysis.

In addition, SonarQube also provides a custom API that allows developers to define their own code smells without
the need of writing the code analysis framework from scratch.
The most common way to extend SonarQube is to develop a plugin which relies on the SonarQube API\@.
In~\cite{sonar_plugin_tutorial}, a comprehensive guide is provided by the SonarQube community which explains how to create
a SonarQube plugin that analyzes source code written in Java.

In short, writing a plugin for SonarQube consists of the following steps:

\begin{flushleft}
    \textbf{Define the rules} -- define a set of rules and implement them using the SonarQube API\@.
    A rule operates on the abstract syntax tree of the program a marks all the tree nodes where it finds
    an issue.
\end{flushleft}

\begin{flushleft}
    \textbf{Create rule loader} -- create a mechanism that will load the plugins and prepare them for the analysis
    execution.
    This usually includes loading descriptions of the rules into memory and registering them with SonarQube.
    Registration allows SonarQube to include the rules in the UI, which in turn allows the end user to enable/disable
    specific rules based on his/her preferences.
\end{flushleft}

\begin{flushleft}
    \textbf{Test the implementation} -- write unit tests using Sonar testing framework to verify that the
    plugin actually detects the correct patterns in the code.
\end{flushleft}

\begin{flushleft}
    \textbf{Deploy the plugin} -- install the plugin to the SonarQube instance.
    This allows the users to run the plugin rules on actual projects in order to detect issues in their source code.
\end{flushleft}

Finally, after all the aforementioned steps have been completed, the user can enable the rules and run static analysis.

Previously we mentioned that SonarQube is often used as a step during CI/CD pipeline.
As reader might know, this is quite late in the development process, and it would be much more beneficial if the issues
would be discovered earlier.
Fortunately, SonarQube provides a solution called SonarLint~\cite{sonarlint}.
SonarLint is an extension for integrated development environment (IDE), which can be synchronized with an instance of SonarQube in order to provide
real-time analysis of the code.
This allows the developers to see and fix the issues before their run and commit their code, which greatly
speeds up the development process.
