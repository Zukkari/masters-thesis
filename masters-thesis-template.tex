% Institute of Computer Science thesis template
% authors: Sven Laur, Liina Kamm
% last change Tõnu Tamme 03.05.2019
%--
% Compilation instructions:
% 1. Choose main language on line 55-56 (English or Estonian)
% 2. Compile 1-3 times to get refences right
% pdflatex bachelors-thesis-template
% bibtex bachelors-thesis-template
%--
% Please use references like this:
% <text> <non-breaking-space> <cite/ref-command> <punctuation>
% This is an example~\cite{example}.

\documentclass[12pt]{article}

% A package for setting layout and margins for your thesis
\usepackage[a4paper]{geometry}

%%=== A4 page setup ===
%\setlength{\paperwidth}{21.0cm}
%\setlength{\paperheight}{29.7cm}
%\setlength{\textwidth}{16cm}
%\setlength{\textheight}{25cm}


% When you write in Estonian then you want to use text with right character set
% By default LaTeX does not know what to do with õäöu letters. You have to specify
% a correct input and font encoding. For that you have to Google the Web
%
% For TexShop under MacOS X. The right lines are
%\usepackage[applemac]{inputenc}
%\usepackage[T1]{fontenc} %Absolutely critical for *hyphenation* of words with non-ASCII letters.
%
% For Windows and Linux the right magic lines are
% \usepackage[latin1]{inputenc}
% \usepackage[latin5]{inputenc}
%
\usepackage[utf8]{inputenc} %standard encoding since 2018 (can be commented out?)
\usepackage[T1]{fontenc} %Absolutely critical for *hyphenation* of words with non-ASCII letters.

% Typeset text in Times Roman instead of Computer Modern (EC)
\usepackage{times}

% Suggested packages:
\usepackage{microtype}  %towards typographic perfection...
\usepackage{inconsolata} %nicer font for code listings. (Use \ttfamily for lstinline bastype)


% Use package babel for English or Estonian
% If you use Estonian make sure that Estonian hyphenation is installed
% - hypen-estonian or eehyp packages
%
%===Choose the main language in thesis
\usepackage[estonian, english]{babel} %the thesis is in English
%\usepackage[english, estonian]{babel} %the thesis is in Estonian


% Change Babel document elements
\addto\captionsestonian{%
    \renewcommand{\refname}{Viidatud kirjandus}%
    \renewcommand{\appendixname}{Lisad}%
}


% If you have problems with Estonian keywords in the bibliography
%\usepackage{biblatex}
%\usepackage[backend=biber]{biblatex}
%\usepackage[style=alphabetic]{biblatex}
%% plain --> \usepackage[style=numeric]{biblatex}
%% abbrv --> \usepackage[style=numeric,firstinits=true]{biblatex}
%% unsrt --> \usepackage[style=numeric,sorting=none]{biblatex}
%% alpha --> \usepackage[style=alphabetic]{biblatex}
%\DefineBibliographyStrings{estonian}{and={ja}}
%\addbibresource{masters-thesis.bib}


% General packages for math in general, theorems and symbols
% Read ftp://ftp.ams.org/ams/doc/amsmath/short-math-guide.pdf for further information
\usepackage{amsmath}
\usepackage{amsthm}
\usepackage{amssymb}

% Optional calligraphic fonts
% \usepackage[mathscr]{eucal}

% Print a dot instead of colon in table or figure captions
\usepackage[labelsep=period]{caption}

% Packages for building tables and tabulars
\usepackage{array}
\usepackage{tabu}   % Wide lines in tables
\usepackage{xspace} % Non-eatable spaces in macros

% Including graphical images and setting the figure directory
\usepackage{graphicx}
\graphicspath{{figures/}}

% Packages for getting clickable links in PDF file
%\usepackage{hyperref}
\usepackage[hidelinks]{hyperref} %hide red (blue,green) boxes around links
\usepackage[all]{hypcap}


% Packages for defining colourful text together with some colours
\usepackage{color}
\usepackage{xcolor}
%\definecolor{dkgreen}{rgb}{0,0.6,0}
%\definecolor{gray}{rgb}{0.5,0.5,0.5}
\definecolor{mauve}{rgb}{0.58,0,0.82}


% Standard package for drawing algorithms
% Since the thesis in article format we must define \chapter for
% the package algorithm2e (otherwise obscure errors occur)
\let\chapter\section
\usepackage[ruled, vlined, linesnumbered]{algorithm2e}

% Fix a  set of keywords which you use inside algorithms
\SetKw{True}{true}
\SetKw{False}{false}
\SetKwData{typeInt}{Int}
\SetKwData{typeRat}{Rat}
\SetKwData{Defined}{Defined}
\SetKwFunction{parseStatement}{parseStatement}


% Nice todo notes
\usepackage{todonotes}

% comments and verbatim text (code)
\usepackage{verbatim}


% Proper way to create coloured code listings
\usepackage{listings}
\lstset{
%language=python,                % the language of the code
    language=C++,
    basicstyle=\footnotesize,        % the size of the fonts that are used for the code
    %numbers=left,                   % where to put the line-numbers
    %numberstyle=\footnotesize,      % the size of the fonts that are used for the line-numbers
    numberstyle=\tiny\color{gray},
    stepnumber=1,                    % the step between two line-numbers. If it's 1, each line
    % will be numbered
    numbersep=5pt,                   % how far the line-numbers are from the code
    backgroundcolor=\color{white},   % choose the background color. You must add \usepackage{color}
    showspaces=false,                % show spaces adding particular underscores
    showstringspaces=false,          % underline spaces within strings
    showtabs=false,                  % show tabs within strings adding particular underscores
    frame = lines,
    %frame=single,                   % adds a frame around the code
    rulecolor=\color{black},           % if not set, the frame-color may be changed on line-breaks within
    % not-black text (e.g. commens (green here))
    tabsize=2,                       % sets default tabsize to 2 spaces
    captionpos=b,                    % sets the caption-position to bottom
    breaklines=true,                 % sets automatic line breaking
    breakatwhitespace=false,         % sets if automatic breaks should only happen at whitespace
    %title=\lstname,                 % show the filename of files included with \lstinputlisting;
    % also try caption instead of title
    keywordstyle=\color{blue},       % keyword style
    commentstyle=\color{dkgreen},    % comment style
    stringstyle=\color{mauve},       % string literal style
    escapeinside={\%*}{*)},          % if you want to add a comment within your code
    morekeywords={*,game, fun}       % if you want to add more keywords to the set
}


% Obscure packages to write logic formulae and program semantics
% Unless you do a bachelor thesis on program semantics or static code analysis you do not need that
% http://logicmatters.net/resources/ndexamples/proofsty3.html <= writing type rules => use semantic::inference
% ftp://tug.ctan.org/tex-archive/macros/latex/contrib/semantic/semantic.pdf
\usepackage{proof}
\usepackage{semantic}
\setlength{\inferLineSkip}{4pt}
\def\predicatebegin #1\predicateend{$\Gamma \vdash #1$}

% If you really want to draw figures in LaTeX use packages tikz or pstricks
% However, getting a corresponding illustrations is really painful


% Define your favorite macros that you use inside the thesis
% Name followed by non-removable space
\newcommand{\proveit}{ProveIt\xspace}

% Macros that make sure that the math mode is set
\newcommand{\typeF}[1] {\ensuremath{\mathsf{type_{#1}}}\xspace}
\newcommand{\opDiv}{\ensuremath{\backslash \mathsf{div}}\xspace}

% Nice Todo box
\newcommand{\TODO}{\todo[inline]}

% A way to define theorems and lemmata
\newtheorem{theorem}{Theorem}


%%% BEGIN DOCUMENT
\begin{document}

%===BEGIN TITLE PAGE
    \thispagestyle{empty}
    \begin{center}

        \iflanguage{english}{%
            \large
            UNIVERSITY OF TARTU\\%[2mm]
            Institute of Computer Science\\
            Computer Science Curriculum\\%[2mm]
        }{%
            TARTU ÜLIKOOL\\
            Arvutiteaduse instituut\\
            Informaatika õppekava\\%[2mm]
        }%\iflanguage

%\vspace*{\stretch{5}}
        \vspace{25mm}

        \Large Stanislav Mõškovski

        \vspace{4mm}

        \huge Building a tool for detecting code smells in Android application code

%\vspace*{\stretch{7}}
        \vspace{20mm}

        \iflanguage{english}{%
            \Large Master's Thesis (30 ECTS)
        }{%
            \Large Bakalaureusetöö (9 EAP)
        }%\iflanguage

    \end{center}

    \vspace{2mm}

    \begin{flushright}
    {
        \setlength{\extrarowheight}{5pt}
        \begin{tabular}{r l}
            \sffamily \iflanguage{english}{Supervisor}{Juhendaja}: & \sffamily Kristiina Rahkema, MSc \\
            \sffamily \iflanguage{english}{Supervisor}{Juhendaja}: & \sffamily Dietmar Pfahl, PhD
        \end{tabular}
    }
    \end{flushright}

%\vspace*{\stretch{3}}
%\vspace{10mm}

    \vfill
    \centerline{Tartu 2020}

%===END TITLE PAGE

% If the thesis is printed on both sides of the page then
% the second page must be must be empty. Comment this out
% if you print only to one side of the page comment this out
%\newpage
%\thispagestyle{empty}
%\phantom{Text to fill the page}
% END OF EXTRA PAGE WITHOUT NUMBER


%===COMPULSORY INFO PAGE
    \newpage

%=== Info in English
    \newcommand\EngInfo{{%
        \selectlanguage{english}
        \noindent\textbf{\large Building a tool for detecting code smells in Android application code}

        \vspace*{3ex}

        \noindent\textbf{Abstract:}

        \noindent
        \TODO{
            Write abstract text here
        }

        \vspace*{1ex}

        \noindent\textbf{Keywords:}\\
        \TODO{List of keywords}
%Layout, formatting, template

        \vspace*{1ex}

        \noindent\textbf{CERCS:}\TODO{CERCS code and name:~\url{https://www.etis.ee/Portal/Classifiers/Details/d3717f7b-bec8-4cd9-8ea4-c89cd56ca46e}}

        \vspace*{1ex}
    }}%\newcommand\EngInfo


%=== Info in Estonian
    \newcommand\EstInfo{{%
        \selectlanguage{estonian}
        \noindent\textbf{\large Building a tool for detecting code smells in Android application code}
        \vspace*{1ex}

        \noindent\textbf{Lühikokkuvõte:}

%\noindent ...

        \TODO{One or two sentences providing a basic introduction to the field, comprehensible to a scientist in
        any discipline.}
        \TODO{Two to three sentences of
        more detailed background, comprehensible to scientists in related disciplines.}
        \TODO{One sentence clearly stating the general problem being addressed by this particular
        study.}
        \TODO{One sentence summarising the main result (with the words ``here we show´´ or their equivalent).}
        \TODO{Two or three sentences explaining what
        the main result reveals in direct
        comparison to what was thought to be the case previously, or how the main result adds to previous knowledge.}
        \TODO{One or two sentences to put the results into a more general context.}
        \TODO{Two or three sentences to provide a
        broader perspective, readily
        comprehensible to a scientist in any
        discipline, may be included in the first paragraph
        if the editor considers that the accessibility of
        the paper is significantly enhanced by their inclusion.}

        \vspace*{1ex}

        \noindent\textbf{Võtmesõnad:}\\
        \TODO{List of keywords}
%Layout, formatting, template

        \vspace*{1ex}

        \noindent\textbf{CERCS:}\TODO{CERCS kood ja nimetus:~\url{https://www.etis.ee/Portal/Classifiers/Details/d3717f7b-bec8-4cd9-8ea4-c89cd56ca46e}}

        \vspace*{1ex}
    }}%\newcommand\EstInfo


%=== Determine the order of languages on Info page
    \iflanguage{english}{\EngInfo}{\EstInfo}
    \iflanguage{estonian}{\EngInfo}{\EstInfo}


    \newpage
    \tableofcontents

    \newpage

    \section{Introduction}\label{sec:introduction}

    \subsection{Research context}\label{subsec:research-context}

    \TODO{
        Describe what code smells are.
        Describe how code smells are different from bugs.
        Shortly about previous research and how we plan to be different.
    }

    \subsection{Research motivation}\label{subsec:research-motivation}

    \TODO{
        Describe why solution proposed in this thesis is useful.
        Goals of the thesis:
        \begin{itemize}
            \item Develop a tool, describe why it would be useful from different perspectives (developers, project managers, data scientists)
            \item Extend the body of knowledge about the occurrence of code smells in Android applications (extend the number of code smells,
            provide analysis results, compare the results with with already published results, additional results for code smells not
            yet published in the literature)
        \end{itemize}
    }

    \subsection{Thesis outline}\label{subsec:thesis-outline}

    \TODO{
        Shortly describe structure of the thesis.
        What does each chapter tell the reader?
    }

    \newpage

    \section{Background}\label{sec:background}

    \subsection{Code smells}\label{subsec:code-smells}
    \TODO{
        Describe code smells in general, what are they, how were they found at first.
        Describe how to fix code smells.
        Describe why would you want to fix them.

        Add all of the implemented code smells into appendix, here we should bring some examples
        about the code smells.

        Bring some examples from the Fowler's list and then also describe
        those that we have implemented.
    }

    \subsection{Related work}\label{subsec:related-work}

    \TODO{
        Describe existing tools.
        Discuss their results and implementations.
        Here we can describe the same 3 tools that were used during the seminar:
        paprika, infusion and anti patterns code smells plugin for SonarQube.
    }

    \subsection{SonarQube}\label{subsec:sonarqube}

    \TODO{
        Describe what is SonarQube.
        Describe why was SonarQube chosen as implementation platform.
        Describe how can SonarQube be exnteded.
        Describe what does it mean to write a plugin for SonarQube: extension points (sensor/rule), what are the possibilities for the user
        (enabling/disabling rules), possibility to run both server side and inside an IDE (SonarLint).
    }

    \newpage


    \section{Method}\label{sec:method}

    \TODO{
    Describe the tool here.
}

\TODO{
    What did we build?

    Plugin for SonarQube that can detect 29 code smells.
    We need a tool that can scan a large number of applications.
    This is best achieved when the tool can be run automatically for a given input project
    and since the corpus is large, the analysis should be performed on the server side by the program
    and not by the human who would perform a manual check.
}
\subsection{SonarQube plugin development}\label{subsec:sonarqube-plugin-development}

In order to fulfil our task of analyzing a large corpus of application to detect the code smells,
we built a tool that is stable, scalable and allows us to aggregate the results of the
analysis in an organized manner.
Moreover, we needed a framework that would allow us to analyze a large corpus of applications programmatically since starting
the analysis manually for every project under observation would be inefficient and unproductive.

For this task, we decided to use the SonarQube platform because it is a de facto tool in the industry to use
for static analysis of the applications.
Not only that, but SonarQube provides possibilities to write custom rules by writing custom plugins.
Since we needed to implement code smells that are not yet defined by the SonarQube, we decided to extend
the tool by writing a plugin that can detect the code smells that are described in subsection~\ref{subsec:code-smells}.

\TODO{
    How?

    Followed tutorial that is available on SonarQube documentation page (https://docs.sonarqube.org/display/PLUG/Writing+Custom+Java+Rules+101).
    But since there were some issues (describe issues with classpath, describe how the analysis works), we had to
    reuse some of the internals of the SonaQube Java module.

    Here we also say that there are multiple contexts where the plugin runs.
    One of the contexts is to run the plugin on the server side, which is supposed
    to be run during CI/CD pipeline, and another context is to run inside developers IDE
    to provide instant feedback without the need to compile the code.
}

To create the plugin, we followed the tutorial provided in the SonarQube documentation~\cite{sonar_plugin_tutorial}.
The documentation provides guidelines on how to create a plugin with custom Java rules, how to test the plugin and
how to register rules with the SonarQube so that it would find them during runtime of the application.
The documentation relies on extension of Sonar Java plugin~\cite{sonar_java_plugin}, which provides an API
for the Java languages abstract syntax tree (AST) and basic interface to create rules, which would be used
during the analysis.

However, this tutorial only focuses on running on an instance of SonarQube and not SonarLint, which is an
extension to run the plugins inside the integrated development environment (IDE).
This is relevant because both SonarQube and SonarLint rely on Sonar compute engine, which means that you can write
a plugin for either of those tools and it would be usable in both of them.

During the runtime of SonarQube, plugins can be installed dynamically, either from the marketplace or they can
also loaded from the \url{/opt/sonarqube/extensions/plugins/} directory as stated in the documentation.
This means that plugins will be loaded dynamically during the execution of the analysis and since the documentation
states the the Sonar Java plugin must be included with \verb|provided| scope during compilation (which means that it
will not be included in the final compiled artifact of the plugin), it means that not all of the classes might be
available at the runtime that were available during compilation.

\TODO{
    What is the goal?

    The goal is to build a tool, that can help developers in static analysis of the code.
    The tool would detect the list of code smells found in the appendix.
    Previously we mentioned that there are 2 contexts in which the plugin can be run, and in
    this thesis we will focus only on the server side of the tool.
    This is because we are interested in analyzing a large corpus of the applications
    and this best done on the server side, because manually skimming through all of
    the detected code smells in the IDE is not an option when you have a corpus of 1000 applications.
}

Previously we mentioned that there are multiple contexts in which the analysis can be run (SonarQube and SonarLint),
however, in this thesis, we will only focus on SonarQube because we are interested in analysis of the large corpus
of applications and it makes no sense to do this manually through the IDE\@.

Thus, we implemented the plugin that detects the code smells described in subsection~\ref{subsec:code-smells} and
can be executed inside a SonarQube instance.

\TODO{
    What tools did we use?

    SonarQube as a platform to run the analysis - provides nice UI for the end users and also
    very mature tool in the industry.
    But most importantly, allows us to analyze the applications and controls the whole flow of the analysis
    (starting the analysis, running the analysis, creating results, uploading them and then displaying them to the user).
    So we only need to provide our custom rules and a way to load them.

    Scala - language that we used to write the rules in.
    Scala is a programming language that combines both functional and object oriented approaches.
    So this was a good choice for me, because I wanted to write the code in a functional way but
    SonarQube is written in Java and API is designed in an object oriented matter.
    So Scala provided nice interop with Java API because it supports both functional and
    imperative approaches like described previously.

    Sonar Java plugin - used this as a base, because it provides all necessary tools that are
    required to parse Java sources into the AST and also provides needed utilities during the analysis
    (detection of cognitive complexity, number of lines of code etc).
}

\subsection{Development tools used}\label{subsec:development-tools-used}

In order to implement the plugin, we used SonarQube as the implementation platform.
We decided to go with SonarQube platform for two primary reasons.
Firstly, SonarQube is a very mature tool in the industry.
Secondly, it controls the flow of the analysis (project configuration, starting
the analysis, creating the results, serializing and parsing the results, uploading them
to the server and displays them to the user), so we can focus on writing the custom code smell
detection rules.
Finally, it provides other features such as a web API, which can be used for exporting the analysis results.

Scala was chosen as the language for the plugin implementation.
We chose Scala, because it combines features of both functional and object oriented languages and since SonarQube
API is written in Java, it allowed for nice interoperability between Java and Scala.
Moreover, Scala provides some features that are not available in Java natively, such as implicit classes and
method parameters, and pattern matching.

As a plugin implementation base, we used Sonar Java plugin since it is recommended
base when writing plugins for the Java language and as mentioned previously, it provides
an API to use Java AST\@.

\subsection{Testing}\label{subsec:testing}

\TODO{
    Describe how did we test it?

    Mostly unit tests, to test the internal architecture of the plugin.
    In order to check the rules validity, SonarQube provides a framework to
    analyze the files the same way that the scanner on the server side in production would do.

    So, the rules were validated by writing a code snippet that would be represents a given
    code smell that we want to detect in a given rule.
    Then, you can describe which line of the code is invalid (e.g. you want to detect classes that contain)
    a code smell pattern.
    Then, the code snippet would contain comment in pattern (provide pattern here and also an example).
}

Once the plugin was implemented, we need to also test it to test the operation of the plugin.
Testing of the plugin can be split into two phases: testing of plugin internal architecture, such as
rule loading, registration, rule metadata loading and testing of rule definitions and verifying that
the rule detects code smells if provided with a valid code snippet.

In order to satisfy the first phase of the testing, we used Scalatest~\cite{scalatest} to write basic tests and to verify
that plugins internal components operate correctly.
We chose Scalatest, because it is a standard unit testing framework for Scala projects and it allows developers
to write unit tests in form of specification.

For the second phase, SonarQube provides a testing framework that allows to test the rules in a similar fashion that
they would have been used in production.
In this framework, testing is performed as follows: firstly, you need to specify a rule which you would like to test by
creating an instance of the rule and passing it to the framework, and secondly, you need to provide a code snippet that
contains a code smell that you would like to verify with the rule.

We can see an example of such test file in figure~\ref{data_class_example}.
This code snippet contains code that a person writing test would consider a code smell for a particular rule
and as seen on line 3, we can specify which line the plugin should report by placing a comment on that line in form of line comment:
\verb|// Noncompliant {{Optional message}}|.
Then the framework will verify if the plugin reports an issue in the provided code snippet and whether the plugin reported
the same line as marked in the source file and if the message was the same as provided in the source file.
This approach allows us to verify the code smells without the need to run the analysis separately on the server and
detect most of the issues with the definitions during testing phase.

\begin{figure} [htb]
    \begin{lstlisting}
package com.example.test;

public class DataClass { // Noncompliant {{Refactor this class so it includes more than just data}}
        public String field1;
        public int field2;
    }
    \end{lstlisting}
    \caption{Example of a data class code smell test file.}
    \label{data_class_example}
\end{figure}



    \subsection{Selected datasets}\label{subsec:selected-datasets}

    \TODO{
        Describe how we selected the dataset.
        We might be using the dataset that was used by the authors of another paper, so that we
        can compare our results to those that they have already provided.
        Also mention here that some of the projects in our corpus were not using the build system,
        so we were not able to analyze them.
        Here we can say that we excluded them because we could build them, but SonarQube itself
        cannot analyze projects that do not use build systems.
    }

    \subsection{Methodology}\label{subsec:Methodology}

    \TODO{
        Describe that we want to perform analysis of projects in order to:
        \begin{itemize}
            \item see how code smells that we implemented are distributed inside analyzed applications
            \item see how our code smells definitions compare to already published results
            \item see how code smells not yet published by the literature are distributed inside analyzed applications
        \end{itemize}

        Here we also need to describe the method that we will use to analyze the applications.
        For example for a each application:
        \begin{enumerate}
            \item Build the project
            \item Analyze the project
            \item Extract what code smells were found / how many were found (statistics)
        \end{enumerate}

        Also for some code smells we need to determine some parameters statistically (using box plot technique).
        This section would need to describe how we would fint hose parameters statistically.
    }

    \newpage


    \section{Results}\label{sec:results}

    \subsection{Developed tools}\label{subsec:developed-tools}

    \TODO{
        Here describe plugin for SonarQube.
        In introduction we mentioned groups that we think the tool might be used for.
        So here we provide our ideas how each group can be helped with our tool,
        provide screenshots or other artifacts that might help our points.

        Also describe the results of developing the bulk analyzer, provide simple
        instructions on how to run this tool, provide output from help command, which
        will show input parameters and basic instruction on how to run.
    }

    \subsection{Analysis results}\label{subsec:analysis-results}

    \TODO{
        Describe the results that we got from project analysis.
        Say how those compare to already existing results.
        Distribution of code smells that are not yet published.
        How many projects in corpus versus how many were actually successfully analyzed.
        Statistics on code smell distributions inside analyzed applications.
    }

    \newpage


    \section{Conclusion}\label{sec:conclusion}

    \TODO{
        Say that we have created a tool and it works on the projects that we checked,
        but we dont know if it actually helps, since we did not perform any empirical study.
        Say that there might be some limitations with the dataset that we have selected.
        Discuss future work.
    }

% BibTeX bibliography
    \bibliographystyle{alpha} %plain=[1], alpha=[BGZ09]
    \bibliography{masters-thesis}

    \addcontentsline{toc}{section}{\refname}


% Use Biblatex if you have problems with Estonian keywords
%\printbibliography %biblatex



    \newpage
%\appendix
%\section*{\appendixname}
    \iflanguage{english}%
    {\section*{Appendix}
        \addcontentsline{toc}{section}{Appendix}
    }%
    {\section*{Lisad}
        \addcontentsline{toc}{section}{Lisad}}


    \section*{I. Glossary}
    \addcontentsline{toc}{subsection}{I. Glossary}

    \newpage

%=== Licence in English
    \newcommand{\licencehint}[2]{\\\hspace*{#1}\textsl(#2)\par}
    \newcommand\EngLicence{{%
        \selectlanguage{english}
        \section*{II. Licence}

        \addcontentsline{toc}{subsection}{II. Licence}

        \subsection*{Non-exclusive licence to reproduce thesis and make thesis public}

        I, \textbf{Stanislav Mõškovski}, %author's name
        \licencehint{10mm}{author's name}

        \begin{enumerate}
            \item
            herewith grant the University of Tartu a free permit (non-exclusive licence) to
            \par
            reproduce, for the purpose of preservation, including for adding to the DSpace digital archives until the expiry of the term of copyright,
            \par
            \textbf{Building a tool for detecting code smells in Android application code}, %
            \licencehint{10mm}{title of thesis}
            \par
            supervised by Kristiina Rahkema and Dietmar Pfahl. %supervisor's name
            \licencehint{10mm}{supervisor's name}
            \item
            I grant the University of Tartu a permit to make the work specified in p. 1 available to the public via the web environment of the University of Tartu, including via the DSpace digital archives, under the Creative Commons licence CC BY NC ND 3.0, which allows, by giving appropriate credit to the author, to reproduce, distribute the work and communicate it to the public, and prohibits the creation of derivative works and any commercial use of the work until the expiry of the term of copyright.
            \item
            I am aware of the fact that the author retains the rights specified in p. 1 and 2.
            \item
            I certify that granting the non-exclusive licence does not infringe other persons' intellectual property rights or rights arising from the personal data protection legislation.
        \end{enumerate}

        \noindent
        Stanislav Mõškovski\\ %author's name
        \textbf{\textsl{dd/mm/yyyy}}
    }}%\newcommand\EngLicence


%=== Licence in Estonian
    \newcommand\EstLicence{{%
        \selectlanguage{estonian}
        \section*{II. Litsents}

        \addcontentsline{toc}{subsection}{II. Litsents}

        \subsection*{Lihtlitsents lõputöö reprodutseerimiseks ja üldsusele kättesaadavaks tegemiseks}

        Mina, \textbf{Alice Cooper}, %author's name
        \licencehint{10mm}{autori nimi}

        \begin{enumerate}
            \item
            annan Tartu Ülikoolile tasuta loa (lihtlitsentsi) minu loodud teose
            \par
            \textbf{Tüübituletus neljandat järku loogikavalemitele}, %title of thesis
            \licencehint{10mm}{lõputöö pealkiri}
            \par
            mille juhendaja(d) on Axel Rose ja May Flower, %supervisor's name(s)
            \licencehint{10mm}{juhendaja nimi}
            \par
            reprodutseerimiseks eesmärgiga seda säilitada, sealhulgas lisada digitaalarhiivi DSpace kuni autoriõiguse kehtivuse lõppemiseni.
            \par
            \item
            Annan Tartu Ülikoolile loa teha punktis 1 nimetatud teos üldsusele kättesaadavaks Tartu Ülikooli veebikeskkonna, sealhulgas digitaalarhiivi DSpace kaudu Creative Commonsi litsentsiga CC BY NC ND 3.0, mis lubab autorile viidates teost reprodutseerida, levitada ja üldsusele suunata ning keelab luua tuletatud teost ja kasutada teost ärieesmärgil, kuni autoriõiguse kehtivuse lõppemiseni.
            \item
            Olen teadlik, et punktides 1 ja 2 nimetatud õigused jäävad alles ka autorile.
            \item
            Kinnitan, et lihtlitsentsi andmisega ei riku ma teiste isikute intellektuaalomandi ega isikuandmete kaitse õigusaktidest tulenevaid õigusi.
        \end{enumerate}

        \noindent
        Alice Cooper\\ %author's name
        \textbf{\textsl{pp.kk.aaaa}}
    }}%\newcommand\EstLicence


%===Choose the licence in active language
    \iflanguage{english}{\EngLicence}{\EstLicence}


\end{document}

