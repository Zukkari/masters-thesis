
To evaluate the plugin and see how it performs, we needed to perform an analysis of existing applications and then
perform statistical analysis to determine if the results are statistically significant.
The analysis itself has two main objectives.
Firstly, we wanted to see how existing (previously implemented  by the literature) code smells were distributed
inside analyzed applications.
Secondly, we wanted to see how new code smells (previously not implemented by the literature) are distributed
inside analyzed applications.


For this purpose, we used a corpus consisting of 1509 applications that were published in~\cite{kotlin_android_corpus} by
\citeauthor{kotlin_android_corpus}.
This corpus consists of Java and Kotlin applications for the Android platform.
In our analysis, we analyzed applications developed in Java and ignored ones implemented in Kotlin.

The main reason why we selected this corpus is that it provides links to the original repositories.
This is important because our tool relies on the analysis of the source code and not compiled \verb|.apk| binaries
that are deployed to the Android platform.
This way we were able to clone the original repositories and perform source code analysis of the applications.
