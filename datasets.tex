\TODO{
    Describe how we selected the dataset.
    We might be using the dataset that was used by the authors of another paper, so that we
    can compare our results to those that they have already provided.
    Also mention here that some of the projects in our corpus were not using the build system,
    so we were not able to analyze them.
    Here we can say that we excluded them because we could build them, but SonarQube itself
    cannot analyze projects that do not use build systems.
}

To evaluate the plugin and see how it performs, we needed to perform an analysis of existing applications.
The analysis itself has two main objectives.
Firstly, we wanted to see how existing (previously implemented by the literature) code smells were distributed
inside analyzed applications.
Secondly, we wanted to see how new code smells (previously not implemented by the literature) are distributed
inside analyzed applications.


For this purpose, we used a corpus consisting of 1509 applications that were published in~\cite{kotlin_android_corpus} by
\citeauthor{kotlin_android_corpus}.
This corpus consists of Java and Kotlin applications for the Android platform.
In our analysis, we analyzed applications developed in Java and ignored ones implemented in Kotlin.


As a result, we were able to analyze 193 applications.
This number is vastly different from the total amount of applications that are present in the corpus for multiple reasons.
Firstly, some of the projects were not using a build system (Gradle or Maven).
This is an issue because to compile the project we would need to specify the classpath manually and
this is quite difficult to do programmatically.
Moreover, even if we managed to compile the projects, SonarQube is provided as a plugin for the build systems which means
that we would still not be able to run the analysis.
Secondly, some of the projects required extra configuration before building, which is also not possible to do
programmatically because each project is different and requires manual configuration.
