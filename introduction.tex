\subsection{Research context}\label{subsec:research-context}

Code smells are patterns in the code that are symptoms of poor design choices.
As opposed to software bugs, code smells do not make the problem run incorrectly but
instead lead to increased fault-proneness and, in the long run, decrease software maintainability.

In~\cite{mannan2016understanding, sonar-plugin-external, paprika-paper} researchers developed
numerous static code analyzers to detect code smells inside Java applications.
However, some of the tools are not available anymore due to being commercial and closed source.
Others are only able to detect a small number of code smells or work with compiled applications instead
of the source files.

In this paper, we developed a tool that is open-source and implements 26 code smells defined by \citeauthor{refactoring-fowler}.
We also performed evaluation of the tool, where we performed an analysis of 240 applications and then
compared our results to already published results in~\cite{mannan2016understanding}.

\subsection{Research motivation}\label{subsec:research-motivation}

\TODO{
Describe why solution proposed in this thesis is useful.
Goals of the thesis:
\begin{itemize}
    \item Develop a tool, describe why it would be useful from different perspectives (developers, project managers, data scientists)
    \item Extend the body of knowledge about the occurrence of code smells in Android applications (extend the number of code smells,
    provide analysis results, compare the results with with already published results, additional results for code smells not
    yet published in the literature)
\end{itemize}
}

\subsection{Thesis outline}\label{subsec:thesis-outline}

\TODO{
Shortly describe structure of the thesis.
What does each chapter tell the reader?
}
