\subsection{Research context}\label{subsec:research-context}

Code smells are patterns in the code that are symptoms of poor design choices.
As opposed to software bugs, code smells do not make the problem run incorrectly but
instead, lead to increased fault-proneness and, in the long run, decrease software maintainability.

In~\cite{mannan2016understanding, sonar-plugin-external, paprika-paper} researchers developed
numerous static code analyzers to detect code smells inside Java applications.
However, some of the tools are not available anymore due to being commercial and closed source.
Others are only able to detect a small number of code smells or work with compiled applications instead
of the source files.

In this paper, we developed a tool that is open-source and implements 26 code smells defined by \citeauthor{refactoring-fowler}.
We also performed an evaluation of the tool, where we performed an analysis of 240 applications and then
compared our results to already published results in~\cite{mannan2016understanding}.

\subsection{Research motivation}\label{subsec:research-motivation}

From the industry perspective, we wanted to develop a tool that would assist different roles in the software development process.
The main focus is the developers, can receive real-time feedback about the quality of the code that they produce.
The project managers can have an overview of the project status and knowledge of any potential vulnerabilities, bugs, or code smells.
The data scientist could look into correlations between various code smells and their occurrences.

From the academic perspective, we wanted to extend the body of knowledge about the occurrence of code smells in Android
applications by extending the number of detected code smells, providing analysis results and then comparing them against
results published in the literature.
Additionally, we wanted to provide an overview of code smells occurrences for the code smells not yet published in the
literature.

\subsection{Thesis outline}\label{subsec:thesis-outline}

The paper consists of 4 main sections.
Section~\ref{sec:background} provides an introduction to code smells, reviews the state of the art implementations
of other static code analyzers, and gives an overview of our implementation platform.
Section~\ref{sec:method} describes the developed tool, tools used for the development, and also walks through
the process of the tool verification procedure.
Section~\ref{sec:results} talks about the results of the paper: the developed tool and the results that we found
during the analysis of the applications.
Section~\ref{sec:discussion} discusses the results of the analysis, compares our results with already published results and
discusses the issues and limitations of our tool.
Furthermore, we discuss the potential usages of the tool for various roles in the software development process.
