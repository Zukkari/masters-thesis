\subsubsection{Tool ``inFusion''}\label{subsec:toolsay}

In~\cite{mannan2016understanding}, \citeauthor{mannan2016understanding} used a tool called \say{inFusion} to analyze a large
corpus of Android and Java desktop applications.
The corpus that authors of~\cite{mannan2016understanding} used consisted of 750 open-source Java applications and 500 Android projects including in total over 22 million lines of code.
According to \citeauthor{mannan2016understanding}, the tool was chosen for multiple reasons.
Firstly, it can detect all the code smells defined by Martin Fowler in~\cite{refactoring-fowler}.
Secondly, this tool has a very high precision (84\%) and recall (100\%) rate (and also F-measure of 91.3\%) according to~\cite{ahmed2015empirical}.
Thirdly, according to the authors of the research, this tool scales to analyzing large applications.
Finally, the tool used to be popular among researchers during the conduction of this study.

The main issue with this tool is that it used to be commercial and is no longer available nowadays.
Due to the tool not being available, we are not able to check the exact code definitions used by it and thus, we cannot
easily replicate the results that were achieved in~\cite{mannan2016understanding}.

Additionally, since the tool was commercial, it would be highly unlikely that the users of the software would
be able to define their code smells to extend the definitions provided by the software vendor.

\subsubsection{Tool ``paprika''}

Paprika is a tool developed by \citeauthor{paprika-paper} in~\cite{paprika-paper}.
This tool analyzes APK files, which is the format in which Android applications are distributed.
One important issue is that this tool analyzes compiled artifacts of the application and not its source code.
It is also worth mentioning that Android applications run on Dalvik virtual machines and not on Java virtual machines.
This is crucial because it is mentioned in~\cite{paprika-paper} the bytecode for those virtual machines
are different and during decoding from one bytecode to another, some information is lost.
This step is required in order to actual Java code representation because some of the code smells
may have been introduced during the compilation process by the compiler and we would like to avoid those.
Losing information during bytecode conversion and code obfuscation together produce quite significant information loss.

According to~\citeauthor{paprika-paper}, paprika analysis consists of four steps.
First, various metrics are extracted from the artifact of the application such as the application name and/or packages.
Paprika builds a model with the following attributes: \texttt{App}, \texttt{Class}, \texttt{Method}, \texttt{Attribute},
\texttt{Variable}, \texttt{ExternalClass}, \texttt{ExternalMethod}.
These entities are linked between each other via different relations to produce a meaningful model that can be
analyzed later.
It is also worth mentioning that the model is built incrementally using the SOOT framework as described in~\cite{paprika-paper}.

Secondly, the existing Paprika model has to be converted into a graph model.
~\citeauthor{paprika-paper} used a graph database to achieve this.
They chose \citet{neo4j} as a database implementation because \citet{neo4j} combined with
Cypher provides great performance for Java applications and can satisfy scalability requirements.

Next, anti-patterns in the code are detected using graph queries.
As mentioned before, the Cypher query language is used to query the database and detect anti-patterns.
Paprika detects the following code smells and anti-patterns: long method, complex class, member ignoring method,
leaking inner class, UI overdraw, heavy broadcast receivers.
The exact code definitions and more detailed descriptions can be found in~\cite{paprika-paper}.

Finally, the software quality score is computed.
According to \citeauthor{paprika-paper}, the score is based on the number of anti-patterns detected.
This score is computed for multiple versions of the same application to see how the number of code smells changes with the evolution
of the application.

To sum up, we can see that this application is built to analyze applications at scale, but it is only limited
to Android applications.
Another disadvantage is that the number of code smells supported is quite small.

\subsubsection{SonarQube plugin ``Anti patterns code smells''}

In~\cite{mannan2016understanding,paprika-paper}, the authors present standalone tools that
were developed from scratch.
In~\cite{sonar-plugin-external}, a plugin for SonarQube has been developed that can detect 19 code and 4
architectural smells.

SonarQube is one of the largest ecosystems that perform static code analysis for a wide variety of languages.
The framework comes with a large number of predefined code smells that are very common in the industry.
In addition, SonarQube also provides a custom API that allows developers to define their own code smells without
the need of writing the code analysis framework from scratch.
SonarQube is open-source, has been in development since 2006 and has over 29,000 commits in its Github repository\cite{sonar-repo}.
Therefore, we can say that this framework is pretty mature.
Furthermore, SonarQube is also one of the most popular tools in industry to perform static code analysis.

Internally the \say{Anti patterns code smells} plugin uses \texttt{DECOR} tool to the detect code smells.
According to~\citeauthor{sonar-plugin-external}, \texttt{DECOR} has 100\% recall rate as well as 80\% precision
for the detection of code smells.
The plugin is able to detect the following code smells: \say{anti-singleton}, \say{base class knows derived class},
\say{base class should be abstract}, \say{blob}, \say{class data should be private}, \say{complex class},
\say{duplicated code}, \say{functional decomposition}, \say{large class}, \say{lazy class},
\say{long method}, \say{long parameter list}, \say{many field attributes but not complex},
\say{message chains}, \say{refused parent bequest}, \say{spaghetti code}, \say{swiss army knife} and
\say{tradition breaker}.
The plugin is also able to detect the following architectural code smells:
\say{unstable dependency}, \say{hub-like dependency}, \say{cyclic dependency},
\say{multiple architectural smells}.
The exact code smell definitions and descriptions can be found in the original paper~\cite{sonar-plugin-external}.

According to~\citeauthor{sonar-plugin-external}, the plugin was able to analyze 103 projects in 35 days
from the Qualitas Corpus Repository~\cite{qualitas-corpus} on a Linux machine with 4 core and 16GB of RAM\@.
