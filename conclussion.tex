
In this thesis we developed a static code analyzer for Java applications in a form
of a plugin for SonarQube plugin.
Then, we performed the analysis of a corpus of applications and analyzed the results.
From the results of the analysis, we could see that that tool actually works since it
detects the code smells that we have defined.
However, we did not perform an empirical study if the tool actually helps the developers to
write better code.

We also understand that there might be some limitations with the dataset that we have selected.
The dataset that we have selected was not previously used for code smell analysis and number of the
projects that we have successfully analyzed is not that big.
Nonetheless, this dataset is enough to verify that the our tool is able to analyze the applications and is able
to do so rather efficiently.

As for the future work, there is still improvement possibilities.
Firstly, we would like to implement nice descriptions for the code smells.
This would allow the developers to see compliant and non-compliant code examples in order to grasp
the idea why a particular code snippet is a code smell.
Secondly, we would like to implement the integration with the IDE, such as IntelliJ IDEA or Eclipse.
This would allow the developers to see the code smells from the comfort of their own IDE and would allow
to detect the code smells during the code writing phase, instead of the compilation time.
